\documentclass[10pt]{article}
\usepackage{amsmath}
\title{On Analysis}
\author{Sachidananda Urs}
\begin{document}
\maketitle

\begin{abstract}
  This article aims to answer two questions, {\em what is analysis?} and {\em why
    do analysis?} These are the notes from Terence Tao's book {\bf Analysis I}.
\end{abstract}

\section*{What is analysis?}
{\em Analysis} is the rigourous study of a subject. In {\em Real Analysis} we
study the qualitative and quantitave behaviour of real numbers, sequences and
series of real numbers, and real valued functions.

In studying real numbers, we are concerned in answering the following questions:
\begin{enumerate}
\item What is a real number?
\item Is there a largest real number?
\item After 0 what is the next real number? [Smallest positive real number]
\item Can a real number be cut into infinitely many pieces?
\item Why does 2 has a square root where as -2 does not?
\item If real and rational numbers are infinite, why are there more reals?
\item How do you take the limit of a sequence of real numbers?
\item Which sequences have limit, which don't?
\item If a sequence can be stopped from escaping to infinity, does it settle
  down and converge?
\item Can infinitely many real numbers can be added and still get a finite real
  number?
\item Can infinitely many rationals be added and still get a non-rational
  number?
\item If the elements of an infinite sum are rearranged, is the sum still the
  same?
\item What does it mean for a function to be continuous?
\item What does it mean for a function to be differentiable?
\item What does it mean for a function to be integrable?
\item What does it mean for a function to be bounded?
\item Can multiple functions be added infinitely?
\item Can you take limits of sequences of functions?
\item Can you differentiate an infinite series of functions?
\item Can you integrate an infinite series of functions?
\item If \( f(0) = 3 \) and \( f(1) = 5 \), does the function take all the
  values between 3 and 5 within its domain \( [0, 1] \) or \( (0, 1) \).
\end {enumerate}



{\em Real Analysis} is the foundation for calculus. Though we have worked on
hundreds of problems in differential and integral calculus, by doing analysis we
can understand what is really happening under the hood.

\section*{Why do analysis?}
There is a philosophical satisfaction in knowing why things work. Also, one can
get into trouble if the rules of calculus are applied without blindly without
knowing where they came from and what the limits of their applicability are.
\break
To prove the above claim, consider the following examples.

\subsection*{Example 1: Division by zero}
Consider the well known cancellation law \( ac = bc \implies a = b \), the
cancellation doesn't work if \( c = 0 \). (Why?)

To answer the question, consider the identity \( 1 \cdot 0 = 2 \cdot 0 \). The
identity is true, however if the cancellation law is applied blindly, we cancel
out the zeroes and end up with \( 1 = 2 \) which is absurd.

In this case the division by 0 is obvious, but it can be more hidden in other
cases.

\subsection*{Example 2: Divergent series}
Consider the geometric series,

\begin{center}
  \( S = 1 + \frac{1}{2} + \frac{1}{4} + \frac{1}{8} + \frac{1}{16} + \cdots \)
\end{center}

Multiply the series by 2 and we get

\begin{center}
  \( 2S = 2 + 1 + \frac{1}{2}+\frac{1}{4}+\frac{1}{8}+\frac{1}{16} + \cdots \) =
  2+S
\end{center}
and \( S = 2 \).

\break
Now apply the same trick to the series

\begin{center}
  \( S = 1 + 2 + 4 + 8 + \cdots \)

  \( 2S = 2 + 4 + 8 + 16 + \cdots \)
\end{center}

Adding 1 to both sides
\begin{center}
  \( 2S + 1 = 1 + 2 + 4 + 8 + 16 + \cdots = S \)
\end{center}

and \( S = -1 \), which is absurd. So, why does the trick work for one series
and not for another?\hfill
\break

Now consider another series
\begin{center}
  \( S = 1 - 1 + 1 - 1 + \cdots \)
\end{center}

We rearrane the series in three different ways and get three different
results. Which one is correct?

\begin{center}
  \( S = 1 - ( 1 + 1 - 1 + 1 - 1 ) \cdots = 1 - S \)
\end{center}
gives  \( \frac {1}{2} \) as the sum of the series.\\

Consider another arrangement
\begin{center}
  \( S = (1 - 1) + (1 - 1) + (1 - 1) \cdots \)
\end{center}
which gives 0 as the sum of the series.\\

Yet another arrangement
\begin{center}
  \( S = 1 + (- 1 + 1) + (- 1 + 1) + (- 1 + 1) \cdots \)
\end{center}
gives 1 as the result of the series.

\subsection*{Example 3: Divergent sequences}
Let x be a real number and let L be the limit

\[ L = \lim_{n\to\infty}x^n \]
Changing variables \( n = m + 1 \) we have

\[ L = \lim_{m+1\to\infty}x^{m+1} = \lim_{m+1\to\infty} x \cdot x^m = x \cdot
\lim_{m+1\to\infty}x^{m} \]
But if \( m + 1 \to \infty \) then \( m \to \infty \), thus we have

\[ \lim_{m+1\to\infty}x^{m} = \lim_{m\to\infty} x^m = \lim_{n\to\infty}x^{n} =
L \]
and thus \( x\cdot L = L \).

At this point we cancel the L's and conclude \( x = 1 \). Or because of our
previous experience of blind cancellation problems we can safely assume that \(L = 0\).
In particular, we seem to have shown

\[ \lim_{n\to\infty}x^{n} = 0 \] for all \(x \ne 1 \).

But is this a legitimate conclusion? It proves to be absurd if we go for
specific values of x. For example for x = 2 our result states \( 1, 2, 4, 8,
\cdots \) converges to 0. And for x = 1 \( 1, -1, 1, -1, \cdots \) converges
0. Which of these conclusions are right?

Analysis helps in understanding these kinds of problems. For more examples refer
{\em Analysis I} by {\em T. Tao}.

\end{document}
