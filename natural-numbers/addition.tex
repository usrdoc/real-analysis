\documentclass[10pt]{article}
\usepackage{amsmath}
\title {Addition Of Natural Numbers}
\author {Sachidananda Urs}
\begin{document}
\maketitle

\begin{abstract}
This document discusses some of the properties of addition of natural
numbers. We state and prove some of the lemmas and propositions related to
addition of natural numbers.
\end{abstract}

\subsection*{Introduction}
We have been trying to build natural numbers axiomatically, as of now we have
the definitiion of natural numbers, increment operation, and a handful of
axioms.

\emph{Isn't this enough?}

With the existing set of definitions we can just increment numbers. How to add
two natural numbers, say 3 and 5? We can achieve by incrementing 5 three times
or maybe incrementing 3, 5 times.

This is an increment more than adding two to five. Which is one increment more
than adding one to five, so on... we can formally rephrase the above discussion
recursively.
\\[4pt]

\textbf{Definition} [\emph{Addition of natural numbers}] Let m be a natural
number. To add 0 to m we define \(0 + m = m\). Now suppose, inductively we have
defined how to add n to m. Then we can add n++ to m by defining \( (n++) + m =
(n + m)++ \)
\\[4pt]
Now elaborating the earlier definition we can write as follows,
\\[4pt]
$ 0 + m = m $, \\
$ 1 + m = (0++) + m = (0 + m)++ = m++ $\\
$ 2 + m = (1++) + m = (1 + m)++ = ((0++) + m)++ = ((0 + m)++)++ = (m++)++ $\\
so on...

%% \raggedright
\begin{flushleft}
\textbf{Note}: The above definition of addition is assymetric because 5 + 3 is
incrementing 3 five times and 3 + 5 is incrementing 5 three times. However, they
yeild same results.
\newpage
\textbf{Lemma}: For any natural number n, $ n + 0 = n $
\\[3pt]
  [Note that this is different from $0 + n = n$ which we defined earlier as n
    and we haven't proved $ a + b = b + a $ yet.]
\\[4pt]
\textbf{Proof}[Proof by induction]

The base case $0 + 0 = 0$ follows since $0 + m = m$ for all natural numbers m
and 0 is a natural number. Now suppose inductively we have proved $n + 0 = n$ we
wish to prove for $n++$. $(n++) + 0 = (n + 0)++$ (By the definition of
addition. And $(n + 0) = n$ by inductive step, hence we have
$ (n++) + 0 = (n + 0)++ = n++$. Thus closes the induction.
\\[6pt]
\textbf{Lemma}: For any natural numbers n and m, $ n + (m++) = (n + m)++ $
\\[3pt]
  [Note that this is similar to earlier lemma, but we are showing that the
    definition of the addition operation is commutative as well. Again, since we
    have not yet showed $ a + b = b + a $ we cannot assume $ (a++) + b = a +
    (b++) = (a + b)++. $ ]
\\[4pt]
\textbf{Proof}: [Proof by induction]

We induct on n keeping m constant. For the base case $n = 0$ we have $ 0 + (m++)
= m++ $ and $ (0 + m)++ = m++ $ thus LHS is equal to RHS, which proves our base
case. Suppose inductively we have proved $ n + (m++) = (n + m)++ $ we wish to
show that this is true for n++ as well. i.e we wish to prove $ (n++) + (m++) =
((n++) + m)++ = ((n + m)++)++. $

Now consider LHS $ (n++) + (m++)$ which is equal to $ (n + (m++))++$, by the
definition of addition and by the inductive step we know $ n + (m++) = (n +
m)++$ and hence we have LHS equal to $((n + m)++)++ $. And consider the RHS
$((n++) + m)++ $ by the defintion of addition we write $((n + m)++)++$ hence we
have LHS equal to RHS which closes the induction.
\\[6pt]
\textbf{Corollary} $n++ = n + 1$
\\[4pt]
\textbf{Proof} We have by earlier lemma $ n + 0 = n$ and $n++$ can be written as
$ (n +0)++$.

Thus we have  $ (n + 0)++ = n + (0++) $ again by earlier lemma we
have $ n + (m++) = (n + m)++ $. We know by definition $ 0++ = 1$ hence we have
$ (n + 0)++ = n + (0++) = n + 1 $.

\end{flushleft}

\subsubsection*{Addition is Commutative}
\textbf{Proposition}: For any natural numbers n and m, $ n + m = m + n $
\\[3pt]
\textbf{Proof}: [\emph{We prove this by induction}

  We induct on n keeping m constant. First the base case, $n = 0$. By the
  earlier lemma we have $m + 0 = m$ and by definition we have $0 + m =m$ hence
  we have $0 + m = m + 0$ which proves the base case.

  Suppose inductively we have proved $ n + m = m + n $. Now we wish to show it
  is true for $n++$, i.e $(n++) + m = m + (n++)$. Consider the LHS $(n++) + m$,
  by the definition of addition $(n++) + m = (n + m)++$. Now consider the RHS
  $m + (n++)$, by the earlier lemma $m + (n++) = (m + n)++$. By the inductive
  step we know $ n + m = m + n$ hence we have the LHS and RHS as $(m + n)++$
  which closes the induction.


\subsubsection*{Addition is associative}
\textbf{Proposition}: For any natural numbers a, b, and c we have $(a + b) + c =
a + (b + c)$
\\[3pt]
\textbf{Proof}:

We induct on a, keeping b and c constant. For the base case we have $a = 0$ and
thus $(0 + b) + c = 0 + (b + c)$ which gives us $b + c = b + c$ which proves our
base case.

Suppose we have proved the proposition inductively i.e $(a+b)+c = a+(b+c)$.

Now we wish to prove the proposition for $a++$.

$((a++)+b)+c = (a++) + (b + c)$

Consider the L.H.S $((a++)+b)+c$

$= ((a+b)++)+c$

$= ((a+b)+c)++$   [By the definition of addition]
\\[3pt]
Now consider the R.H.S $(a++)+(b+c)$

= $(a+(b+c))++$

And we have by the inductive step and by axiom 4 $((a+b)+c) = (a+(b+c))$. Which
closes the induction.
\\[6pt]
\textbf{Proposition}: \emph{Cancellation Law}

Let a, b, c be natural numbers such that $a + b = a + c$ then $ b = c $.
\\[3pt]
\textbf{Proof}

[Note that we cannot use subtraction or negative numbers to prove this. We prove
  this by induction, we induct on a, and prove that LHS = RHS]

First let us consider the base case $a = 0$, we have $0 + b = 0 + c$ by earlier
propositions we have $ b = c $, which proves our base case. And suppose
inductively we have the the cancellation law for a (i.e $a + b = a + c$ implies
$b = c$); now we wish to prove the cancellation law for $a++$.

That is we wish to show $(a++) + b = (a++) + c$ implies $b = c$. By the
definition of addition we have $(a++) + b = (a + b)++$ and $(a++) + c = (a+c)++$
and thus $(a + b)++ = (a + c)++$. By \emph{\textbf{Axiom 4}} we have $a + b = a
+ c$ and by the inductive step $b = c$, which closes the induction.

\subsubsection*{Addition and positivity}
\textbf{Definition} \emph{Positive Natural Numbers}

A natural number is said to be positive iff it is not equal to 0.
\\[6pt]
\textbf{Proposition}
If \emph{a} is positive and {\em b} is a natural number then $a + b$ is
positive.
\\[3pt]
\textbf{Proof}: [\emph{Proof by induction}]

We induct on b (because we need 0?) keeping a constant. For the base case
$b =0$, we have $a + 0 = a$ (by earlier lemma) which is positive, this proves
the base case. Now suppose inductively we have proved $a + b$ is positive. We
wish to prove for b++, i.e $a + (b++) = (a + b)++$ is positive.
Now by \emph{\textbf{Axiom 3}} $a + (b++) = (a + b)++$ cannot be 0, since 0 is
not a successor of any natural number. And by inductive step $(a+b)$ is positive
and more so $(a + b)++$ is positive, which closes the induction.
\\[6pt]
\textbf{Corollary}: If a and b are natural numbers such that $a + b = 0$, then
$a = 0$ and $ b = 0$
\\[3pt]
\textbf{Proof}: [We cannot use induction in this case, we use the earlier
  proposition.]

Suppose for the sake of contradiction $ a \ne 0$ or $b \ne 0$. If $a \ne 0$ then
$a + b$ is positive (by proposition) and if $b \ne 0$ then $a+b$ is again
positive, a contradiction in both cases.

Hence if $a+b = 0$ then both a and b are equal to 0. If either one of them is
positive then $a+b$ is not equal to 0.
\\[6pt]
\textbf{Lemma}: Let a be a positive number. Then there exists exactly one
natural number b such that $b++ = a$
\\[3pt]
\textbf{Proof}

For the sake of contradiction let there exist two natural numbers b and c such
that $b++ = a$ and $c++ = a$. By axiom 4 we have $b = c$, hence there exists
exactly one natural number such that $b++ = a$.

\subsubsection*{Notion of Order}
\textbf{Definition}[\emph{Ordering of the Natural Numbers}]

  Let n and m be natural numbers, we say that n is greater than or equal to m,
  and write $n \geq m$ or $m \leq n$, if we have $n = m + a$ for some natural
  number a. We say n is strictly greater than m and write $n > m$ or $m < n$
  if and only if $n \geq m$ and $n \neq m$
  \\[6pt]
  Also note that $n++ > n$ for any n; thus there is no largest natural number n,
  because next number $n++$ is always larger still.

  The definition of ordering of natural numbers is beautiful. Or how else would
  you say that a number is greater than another?
\\[6pt]
\textbf{Proposition} [\emph{Basic Properties of order for natural numbers}]

  Let \emph{a, b, c} be natural numbers then:

  \begin{enumerate}
  \item Order is reflexive: $a \geq a$
  \item Order is transitive: if $a \geq b$ and $b \geq c$ then $a \geq c$
  \item Order is anti-symmetric: if $a \geq b$ and $b \geq a$ then $a=b$
  \item Addition preserves order: if $a \geq b$ iff $a+c \geq b+c$
  \item $a < b$ iff $a++ \leq b$
  \item $a < b$ iff $b=a+d$ for some positive number d
  \end{enumerate}

\textbf{Order is reflexive}: $a \geq a$

By the definition of ordering we have $a \geq a$ iff $a = a+b$ for some natural
number b. For $b=0$ we have $a= a+0$ and b is a natural number which implies $a
\geq a$.

\textbf{Order is transitive}: if $a \geq b$ and $b \geq c$ then $a \geq c$

By the definition of ordering if $a \geq b$ then $a=b+j$ for some natural number
j.

And $b \geq c$ then $b=c+k$ for some natural number k.

And we can write $a=b+k$ as $a=c+j+k$ since j and k are natural numbers, by
axiom 2, $j+k$ is a natural number.

Hence it follows $a \geq c$

\textbf{Order is anti-symmetric}: if $a \geq b$ and $b \geq a$ then $a=b$

If $a \geq b$, then by definition $a = b+c$ for some natural number c.

If $b \geq a$, then again by definition $b=a+d$ for some natural number d.

By adding the above $a+b = b+c+a+d$ and $a+b = b+a+c+d$ (addition is
commutative).

And again $a+b=a+b+c+d$ and we can write the L.H.S as $(a+b)+0 = (a+b)+c+d$ and
by the cancellation law we have $c+d=0$.

By the earlier proposition if $c+d=0$, then $c=0$ and $d=0$.

Thus we have $a=b+c$ and since $c=0$ we have $a=b+0$ or $a=b$. We can show $b=a$
similarly.

\textbf{Addition preserves order}: if $a \geq b$ iff $a+c \geq b+c$

We have to prove that (i) if $a \geq b$ then $a+c \geq b+c$ and (ii) if $a+c
\geq b+c$ then $a \geq b$.

Case (i): If $a \geq b$, then by definition $a=b+x$ for some natural number x
and $a+c = b+x+c$, $a+c = b+c+x$, and $a+c = (b+c)+x$ which implies $a+c \geq
b+c$.

Case (ii): If $a+c \geq b+c$ by definition $a+c = b+c+x$ and $a+c = b+x+c$ and
by the cacellation law we have $a = b+x$ for some natural number x.

And thus proves $a \geq b$ 

\flushleft
\newpage
\textbf{Proposition} [\emph{Trichotomy of order for natural numbers}]
Let {\em a} and {\em b} be natural numbers, then exactly one of the following
statements is true: $a < b$, $a = b$ or $a > b$

[\emph{The proof is tricky, we first prove that we cannot have more than one of
    the claim is true at the same time and then we prove at least one of the
    statements is true. This is unconventional (at least for me).}]
\\[5pt]
Firstly we show that we cannot have more than one statement $a < b$, $a = b$
or $a > b$ holding at the same time. (Note: we do not try to show only one of
the claim is true, we can't).
\\[3pt]
If $a < b$, then by definition $a \ne b$. Again if, $a > b$, then $a \ne b$.

And if $a>b$ and $a<b$ then $a = b$ (By above proposition)

Thus no more than one of the statement is true.
\\[4pt]
Now we prove that at least one of the statements is true (we use induction for
that). We keep b fixed and induct on a.

Consider the base case $a = 0$, if $a=0$ then $a \leq b$ for all b, because by
\emph{Axiom 3}, 0 is not a successor of any natural number. Since b is a natural
number we have $b=0$ or $0<b$, which proves the base case.
\\[4pt]
Suppose we have proved the proposition inductively for a, we wish to prove it
for $a++$

From the trichotomy for a, there are three cases $a < b$, $a = b$ or $a > b$
\begin{enumerate}
\item If $a>b$, then $a++ > b$ because we have by earlier proposition, addition
  preserves order. i.e $a>b$ then $a++ > b++$ and more so $a++ > b$.

\item If $a = b$, then $a++ > b$ because, we know by the earlier definition $a++
  > a$ for every natural number a, and we have $a=b$, hence $a++>a=b$ or $a++>b$

\item Now suppose that $a<b$, then by earlier proposition $a++ \leq b$. Thus
  $a++<b$ or $a++=b$ in either case we are done.
\end{enumerate}
This closes the induction.

\end{document}
