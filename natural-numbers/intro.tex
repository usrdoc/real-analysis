\documentclass[10pt]{article}
\usepackage{amsmath}
\title{Introduction to Natural Numbers}
\author{Sachidananda Urs}
\begin{document}
\maketitle

\begin{abstract}
This document tries to answer the questions ``\emph{What is the motivation to
study the concept of Natural numbers?}'' and ``\emph{why do we analyse their
properties?''}
\end{abstract}

\section*{Why study Natural Numbers?}
We have dealt with numbers and manipulated the rules of algebra all our
lives. By going back to basics we understand why the rules work at all. For
example we know $ a(b+c) = ab + ac $ and $ab = ba$. This is seemingly an obvious
rule, we will study why this is so obvious. Even though the rules are obvious,
the proof is not really easy.

The different number systems are $ N \to Z \to Q \to R \to C $, in the order of
complexity, where each is used to build the next one.

We will learn how to define the natural numbers (\textbf{N}) using Peano axioms,
and use naturals to define integers, and integers are used to define rationals,
and finally we use rationals to define reals.
\\[6pt]
\textbf {Definition [Informal]}

A natural number is any element of the set $ N~=~\{0, 1, 2, 3, ... \} $
The set contains all the numbers starting from 0 and then counting forward
indefinitely.

The natural number system can be defined using two fundamental concepts; the
number 0, and the increment operation.

With these simple operations we define complicated operations:
\begin{itemize}
\setlength{\itemsep}{0pt}
\item[] addition --- repeated increments
\item[] multiplication --- repeated addition
\item[] exponentiation --- repeated multiplication
\end{itemize}

We omit subtraction and division as they are not closed under natural numbers.
\\[6pt]
\textbf{On Peano Axioms}

To define the natural numbers we use five axioms called the Peano axioms. It is
the standard way to define natural numbers.
However, this is not the only way to define the natural numbers. Cardinality of
finite sets can also be used, but Peano axioms are the standard and widely used
method to define natural numbers.

\begin{itemize}
\setlength{\itemsep}{0pt}
\item[] \textbf{Axiom 1} 0 is a natural number.
\item[] \textbf{Axiom 2} If n is a natural number n++ is also a natural number.
\item[] \textbf{Axiom 3} 0 is not the successor of any natural number. i.e if n is a natural
number n++ != 0.
\item[] \textbf{Axiom 4} Different natural numbers have different successors i.e if n and m are
natural numbers and n != m then n++ != m++. Equivalently if n++ = m++ then n = m.
\item[] \textbf{Axiom 5} \emph{[Principle Of Mathematical Induction]} Let p(n) is any property pertaining
to a natural number n. Suppose that p(0) is true, and suppose that whenever p(n)
is true, p(n++) is also true. Then P(n) is true for every natural number.
\end{itemize}

\begin{flushleft}
\textbf{\large{Propositions}}
\end{flushleft}

\textbf{Proposition:} 3 is a natural number
\\[4pt]
\emph{How to prove it?} We make use of two Peano axioms \emph{(1 and 2)}

0 is a natural number \emph{(axiom 1)}. By axiom 2, 1 := 0++ is a natural
number, 2 := 1++ is a natural number and applying axiom 2 again 3 := 2++ is a
natural number. Thus 3 is a natural number.
\\[4pt]
\textbf{Discussion:} \emph{Are axioms 1 and 2 enough to define natural numbers?}
Not exactly, consider the possibility of wrapping around. i.e after 0, 1, 2, 3
what if 3++ = 0. Axiom 1 and 2 does not prevent this possibility. Axiom 3 (see
above) prevents this possibility.
\\[4pt]
\textbf{Proposition:} 4 is not equal to 0

By repeated application of axiom 1 and 2 we have 0, and 4~:=~((((0++)\-++)++)++)
i.e 4~:=~3++. By axiom 3, 0 is not a successor of any natural number. Which
implies $3++ \ne  0$ Or $4 \ne 0$.

Now again repeating the question \emph{``Are axioms 1, 2, and 3 enough to define
the natural number system?''}

Answer is still no. This scheme does not rule out the possibility that the
natural number stops at an arbitrary number. For example, 3++~=~4, 4++~=~4, 5++~
=~4, so on\dots i.e the natural number system hits the ceiling.
\\[4pt]
Axiom 4 prevents this case from happening.
\\[4pt]
\textbf{Proposition:} 6 is not equal to 2.

Can axiom 4 alone prove this proposition? 2 := 1++ and 6 := 5++. Since 1 and 5
are natural numbers and different, they have different successors, hence 2 !=
6. i.e 1 != 5 hence 2 != 6.
This proof has a flaw, we assumed 1 and 5 are different without actually proving
that fact. However if we can prove 1 and 5 are different, the proof will be
complete.

Consider the alternate proof (\emph{proof by contradiction}). Suppose 6~=~
2, then 5++~=~1++. Then by axiom 4, 5 = 1 and thus 4++~=~0++, thus 4~=~0. But by
our earlier proposition $4 \ne 0$, hence $6 \ne 2$.
\\[6pt]
\textbf{\large{Notes}}
\begin{enumerate}
\item The Peano axioms define natural numbers.
\item Each individual number is finite, the set of natural numbers is infinite.
\item $\infty$ is not one of the natural numbers. Calling $\infty$ as the
      largest natural number is wrong.
\item The natural numbers can approach $\infty$ but never actually reach it. That
   is, N is infinite but consists of individually finite elements.
\item However, there are other number systems which ``admit'' infinite numbers:
\begin{enumerate}
  \item Cardinals
  \item Ordinals
  \item p-adics
\end{enumerate}
\end{enumerate}

\end{document}
